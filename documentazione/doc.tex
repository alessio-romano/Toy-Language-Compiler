% !TEX encoding = UTF-8 Unicode
% !TEX TS-program = pdflatex
% !TeX spellcheck = it-IT
% Preambolo
\documentclass[a4paper,12pt]{article}
\pagestyle{plain}
\usepackage[T1]{fontenc}
\usepackage[utf8]{inputenc}
\usepackage{geometry}
\usepackage{amssymb}
\usepackage{amsmath} % per i simboli matematici
\geometry{a4paper,top=2.5cm,bottom=2.5cm,left=2cm,right=2cm,heightrounded,bindingoffset=5mm}
\usepackage{microtype} % per migliorare il riempimento delle righe
\usepackage{setspace} % per l'interlinea
\onehalfspacing
\usepackage{xcolor} % per i colori
\usepackage{booktabs} % per le tabelle
\usepackage{multirow}
\usepackage{caption}
\usepackage{subfig}
% per i link
\usepackage{hyperref}
\hypersetup{
  colorlinks   = true,    % Colours links instead of ugly boxes
  urlcolor     = blue,    % Colour for external hyperlinks
  linkcolor    = black,    % Colour of internal links
  citecolor    = red      % Colour of citations
}
\usepackage[english, italian]{babel}
\newcommand\resw[1]{\mathtt{#1}}
\newcommand\mi[1]{\mathit{#1}}
%Comandi per Grammatica
\newcommand\nonterm[1]{\texttt{\textcolor{violet}{#1}}}
\newcommand\term[1]{\texttt{\textcolor{teal}{#1}}}
\newcommand\production[1]{\texttt{#1} \texttt{:=}}
\newcommand\emptyprod{\texttt{\textcolor{teal}{/* empty */}}}
\newcommand\state[1]{\textcolor{violet}{\texttt{#1}}}
\title{Progetto Compilatori \\ \normalsize{A.A. 2020/2021}}
\author{Gaetano Antonucci \and Alessio Romano}
\date{29 Gennaio 2021}
\begin{document}
    \maketitle
    \tableofcontents
    \newpage

	\section{Scelte progettuali}
	\subsection{Analizzatore Lessicale - JFlex}
	
	
	Per gestire gli errori "Commento non chiuso" e "Stringa costante non completata", sono stati utilizzati gli stati in JFlex:
	\state{COMMENT} e \state{COMMENT2} per i commenti, \state{STRING} per le stringhe costanti per rendere più agevole la rilevazione degli errori.\newline

	Il procedimento per i commenti è il seguente:
	\begin{itemize}
    	\item Si accede allo stato \state{COMMENT} all'inizio di un commento (/*).
        \item Se nello stato \state{COMMENT} si legge un asterisco (*) si passa nello stato \state{COMMENT2} poichè potrebbe corrispondere all'asterisco
		 		che precede il simbolo di fine commento
		\item Se in \state{COMMENT2} si legge il simbolo di fine commento (/) si ritorna nello stato iniziale \state{YYINITIAL}. In tutti gli altri casi si torna in \state{COMMENT}
		\item Se si raggiunge la fine del file (\state{EOF}) mentre ci si trova in \state{COMMENT} o \state{COMMENT2} viene generato l'errore "Commento non chiuso"
	\end{itemize}

	Il procedimento per le stringhe è il seguente:
	\begin{itemize}
	\item Si accede allo stato \state{STRING} quando si leggono i doppi apici (")
	\item All'interno dello stato, tutto i caratteri letti vengono aggiunti in una variabile di tipo StringBuffer che rappresenta la stringa costante in questione
	\item Sono stati gestiti in maniera specifica tutti i caratteri di escape (e.g. \verb!\n!, \verb!\t!, \verb!\r!) in modo da
            inserirli letteralmente nella variabile precedentemente menzionata
	\item A causa della naturale presenza dei caratteri di escape nel sorgente analizzato, si sono dovuti distinguere i caratteri di escape espliciti, inseriti in una stringa costante,  				da quelli impliciti nel sorgente i quali dovranno essere ignorati
	\end{itemize}

   Nelle precedenti esercitazioni (es3 e es4) si era deciso di non permettere che i numeri in virgola mobile potessero avere solo zeri dopo la virgola (e.g. 34.0).

   In un secondo momento ci si è accorti che, in molti linguaggi di programmazione, quali il C, è possibile utilizzare numeri nella precedente forma in maniera tale da 
   usare una rappresentazione in virgola mobile di un numero intero per evitare troncamenti nelle operazioni.
   Di conseguenza, l'espressione regolare per il riconoscimento dei numeri in virgola mobile è stata semplificata.

	
	\subsection{Analizzatore Sintattico - CUP}

	\subsubsection{Grammatica Utilizzata}
	La grammatica fornita nelle specifiche del linguaggio Toy è stata modificata accertandosi di non modificare il linguaggio generato.
	Sono stati introdotti due \emph{nuovi} non-terminali, \textcolor{blue}{ProcBody} e \textcolor{blue}{ParIdList}
	in modo da semplificare le produzioni di Proc, in particolare:
	
    \begin{itemize}
	\item \textcolor{blue}{ProcBody} è stato aggiunto per semplificare lo sviluppo del parser tramite CUP
	\item \textcolor{blue}{ParIdList} (le cui produzioni sono equivalenti a quelle di IdList) è stato aggiunto perché è stato deciso che gli elementi della tabella dei simboli
            venissero creati dal parser. Conseguentemente, è necessario distinguere il tipo di lookup da effettuare nella tabella dei simboli nel caso dei parametri
            di una procedura (lookup fatta solo nello scope corrente creato dalla procedura e non in tutti gli scope).
	\end{itemize}
	La grammatica risultante è la seguente:\newline

	\begin{tabular}{rl}
    \production{Program}     & \nonterm{VarDeclList} \nonterm{ProcList} \\
	\production{VarDeclList} & \nonterm{VarDecl} \nonterm{VarDeclList} \\
	                                     | & \emptyprod \\
	\production{VarDecl}      & \nonterm{Type} \nonterm{IdListInit} \term{SEMI} \\
	\production{Type}           & \term{INT} \\ 
					   | &\term{BOOL} \\ 
					   | &\term{FLOAT} \\
					   | &\term{STRING} \\
	\production{IdListInit}     &\term{ID} \\
					   |&\nonterm{IdListInit} \term{COMMA} \term{ID} \\
					   |&\term{ID} \term{ASSIGN} \nonterm{Expr} \\
					   |&\nonterm{IdListInit} \term{COMMA} \term{ID} \term{ASSIGN} \nonterm{Expr} \\
	\production{ProcList}    &\nonterm{Proc} \\
					  |&\nonterm{Proc} \nonterm{ProcList} \\
	\production{Proc}         & \term{PROC} \term{ID} \term{LPAR} \nonterm{ParamDeclList} \term{RPAR} \nonterm{ResultTypeList} \term{COLON} %
						\nonterm{ProcBody}\\
					  |& \term{PROC} \term{ID} \term{LPAR} \term{RPAR} \nonterm{ResultTypeList} \term{COLON} \nonterm{ProcBody}\\
	\production{\textcolor{blue}{ProcBody}} & \nonterm{VarDeclList} \nonterm{StatList} \term{RETURN} \nonterm{ReturnExprs} \term{CORP} \term{SEMI} \\
	 				  |& \nonterm{VarDeclList} \term{RETURN} \nonterm{ReturnExprs} \term{CORP} \term{SEMI} \\
	\production{ParamDeclList} & \nonterm{ParDecl} \\
						  |& \nonterm{ParamDeclList} \term{SEMI} \nonterm{ParDecl} \\
	\production{ParDecl}            & \nonterm{Type} \nonterm{ParIdList} \\
	\production{\textcolor{blue}{ParIdList}}           & \term{ID} \\
						  |& \nonterm{ParIdList} \term{COMMA} \term{ID} \\
	\end{tabular}

	\begin{tabular}{rl}
	\production{ResultTypeList} & \nonterm{ResultType} \\
						  |& \nonterm{ResultType} \term{COMMA} \nonterm{ResultTypeList} \\
	\production{ResultType}       & \nonterm{Type} \\
						  |& \term{VOID} \\
	\production{StatList} 		   & \nonterm{Stat} \term{SEMI} \\
						  |& \nonterm{Stat} \term{SEMI} \nonterm{StatList} \\
	\production{Stat}                  & \nonterm{IfStat} \\
						  |& \nonterm{WhileStat} \\
						  |& \nonterm{ReadlnStat} \\
						  |& \nonterm{AssignStat} \\
						  |& \nonterm{CallProc} \\
    \production{IfStat}                   & \term{IF} \nonterm{Expr} \term{THEN} \nonterm{StatList} \nonterm{ElifList} \nonterm{Else} \term{FI} \\
    \production{ElifList}                 & \nonterm{Elif} \nonterm{ElifList} \\
    						|& \emptyprod \\
   \production{Elif}			 & \term{ELIF} \nonterm{Expr} \term{THEN} \nonterm{StatList} \\
   \production{Else}   			 & \term{ELSE} \nonterm{StatList} \\
   						|& \emptyprod \\
   \production{WhileStat}             & \term{WHILE} \nonterm{StatList} \term{RETURN} \nonterm{Expr} \term{DO} \nonterm{StatList} \term{OD} \\
   						|& \term{WHILE} \nonterm{Expr} \term{DO} \nonterm{StatList} \term{OD} \\
  \production{ReadlnStat}           & \term{READ} \term{LPAR} \nonterm{IdList} \term{RPAR} \\
  \production{IdList}                    & \term{ID} \\
					        |& \nonterm{IdList} \term{COMMA} \term{ID} \\
  \production{WriteStat}              & \term{WRITE} \term{LPAR} \nonterm{ExprList} \term{RPAR} \\
  \production{AssignStat}           & \nonterm{IdList} \term{ASSIGN} \nonterm{ExprList} \\
  \production{CallProc}         & \term{ID} \term{LPAR} \nonterm{ExprList} \term{RPAR} \\
						|& \term{ID} \term{LPAR} \term{RPAR} \\
  \production{ReturnExprs}    & \nonterm{ExprList} \\
						 |& \emptyprod \\
	\production{ExprList}           & \nonterm{Expr} \\
						 |& \nonterm{Expr} \term{COMMA} \nonterm{ExprList} \\
   \end{tabular}

  \begin{tabular}{rl}
   \production{Expr}                    & \term{NULL} \\
						|& \term{TRUE} \\
						|& \term{FALSE} \\
						|& \term{INT\_CONST} \\
						|& \term{FLOAT\_CONST} \\
						|& \term{STRING\_CONST} \\
						|& \term{ID} \\
    				                 |& \term{MINUS} \nonterm{Expr} \\
						|& \nonterm{Expr} \term{PLUS} \nonterm{Expr} \\
						|& \nonterm{Expr} \term{MINUS} \nonterm{Expr} \\ 
						|& \nonterm{Expr} \term{TIMES} \nonterm{Expr} \\
						|& \nonterm{Expr} \term{DIV} \nonterm{Expr} \\
    						|& \term{NOT} \nonterm{Expr} \\
						|& \nonterm{Expr} \term{AND} \nonterm{Expr} \\
						|& \nonterm{Expr} \term{OR} \nonterm{Expr} \\
						|& \nonterm{Expr} \term{GT} \nonterm{Expr} \\
						|& \nonterm{Expr} \term{GE} \nonterm{Expr} \\
						|& \nonterm{Expr} \term{LT} \nonterm{Expr} \\
						|& \nonterm{Expr} \term{LE} \nonterm{Expr} \\
						|& \nonterm{Expr} \term{EQ} \nonterm{Expr} \\
						|& \nonterm{Expr} \term{NE} \nonterm{Expr} \\
						|& \nonterm{CallProc} \\
    \end{tabular}

	\subsubsection{Parser}
	In fase di progettazione del compilatore, già a partire dall'analizzatore sintattico, si è deciso di gestire la gerarchia di tabelle dei simboli tramite uno stack.
	Sarà quindi il parser a creare le singole tabelle e a collegarle tra di loro ogni volta che un costrutto del linguaggio definisce un nuovo scope.
	Di conseguenza, nelle azioni semantiche delle produzioni dei non terminali che definiscono un nuovo scope, si è inserito il codice per la gestione
	dello stack di tabelle.
	
	\subsubsection{Generazione Abstract Syntax Tree}

 	Per la gestione delle componenti dell'albero sintattico, si è optato per l'utilizzo di una classe padre
	\texttt{Node.java} contenente l'attributo nome in quanto comune a tutte le componenti.
    Ogni classe rappresentante una componente diversa dell'albero, quindi, estende la classe \texttt{Node.java}.

	Ragionamento analogo al punto precedente, è stato fatto per le classi che rappresentano gli Statement del
    linguaggio di programmazione Toy. In questo caso, la classe padre è \texttt{StatOp.java}.

   Siccome ci sono alcune componenti dell'albero sintattico che sono comuni a più produzioni della grammatica,
   si è scelto di avere, \nonterm{Expr} e \nonterm{IdListInit} come interfacce. Questo perché, ad esempio, \term{ID} (che corrisponde ad
   una foglia dell'albero) può essere sia visto come \nonterm{Expr} sia come un elemento di \nonterm{IdListInit}.
   Ciò è stato fatto per ottenere, ad esempio, una lista di \nonterm{Expr} i cui elementi sono eterogenei (e anche perché in
   Java non esiste l'ereditarietà multipla).
   Per applicare il pattern Visitor, si è utilizzata l'interfaccia \texttt{Visitor.java} che, per nostra scelta,
   contiene le firme distinte dei vari metodi \texttt{visit()}, ognuno dei quali è specifico per una componente
   dell'albero sintattico.

   Una visualizzazione in XML dell'AST è stata ottenuta tramite un visitor (\texttt{XmlVisitor.java}) 
   Per la generazione del file xml si è utilizzata la libreria jdom (v 2.0.6) inserita come external library

	\subsection{Analizzatore Semantico}
	L'analisi semantica viene effettuata utilizzando due visite dell'AST:
	\begin{itemize}
		\item La prima serve ad aggiungere informazioni alla symbol table necessarie per il type checking. In particolare, vengono aggiunti:
			\begin{itemize}
				\item I tipi delle variabili all'interno dello specifico scope;
				\item I tipi di ritorno delle procedure (eventualmente multipli) nello scope del chiamante;
				\item I tipi dei parametri delle procedure all'interno dello scope della procedura;
			\end{itemize}

		\item La seconda esegue il type checking, e altri controlli semantici, quali i seguenti:
			\begin{itemize}
				\item Controllo della dichiarazione di una singola procedura main;
				\item Controllo del tipo di ritorno della procedura main (deve avere forma \texttt{proc main() void});
				\item Controllo degli utilizzi di procedure non dichiarate all'interno del codice;
				\item Controllo degli utilizzi di variabili non dichiarate all'interno del codice;
				\item Controllo dei parametri della procedura write (non accetta valori null);
			\end{itemize}
	\end{itemize}
	
	Nota: A differenza di quanto indicato nelle specifiche, è stata utilizzata la forward reference (fref) per le procedure. Di conseguenza, non è strettamente necessario dichiarare ogni procedura prima del suo uso\newpage

	\subsection{Generazione del codice intermedio}

	\subsubsection{Gestione dei valori di ritorno multipli per una procedura}\label{cg:procMultipleRet}
		In fase di progettazione del compilatore si è deciso di gestire i valori di ritorno multipli per una procedura nel seguente modo:
			\begin{enumerate}
				\item Il tipo di ritorno nella firma della funzione in C corrisponderà al primo valore di ritorno nella firma della procedura in Toy.
                        Questa scelta è dovuta al fatto che il linguaggio C non permette l'utilizzo di valori di ritorno multipli per una funzione.
				\item Per i valori di ritorno successivi al primo, vengono definite delle variabili globali
						i cui nomi sono nella forma "procedureName\_number" dove number parte da 1.
				\item Si assegnano i valori di ritorno successivi al primo alle variabili definite al punto precedente.
				\item Si aggiunge la clausola \texttt{return} seguita dal primo valore di ritorno della corrispondente procedura in Toy.
			\end{enumerate}

	\subsubsection{Gestione di assegnazioni multiple}

		In fase di progettazione del compilatore si è deciso di gestire le assegnazioni multiple secondo il seguente procedimento:
		Si distinguono due possibili casi:

		\begin{enumerate}
			\item Nel lato destro dell'assegnazione multipla non sono presenti chiamate a procedure con valori di ritorno multipli
			\item Nel lato destro dell'assegnazione multipla sono presenti chiamate a procedure con valori di ritorno multipli
		\end{enumerate}
		
		In entrambi i casi, l'assegnazione multipla in Toy è la seguente: $var_1, \dots, var_n\, := \, val_1, \dots, val_n;$ 			
		Il codice C generato, è ottenuto rispettivamente come segue:
		\begin{enumerate}
			\item Nel primo caso, il codice generato, è $var_i = val_i; \, \forall i \in \{1, \dots, n\}$
					
			\item	Nel secondo caso, in Toy, è possibile che i valori $val_i$ nella parte destra dell'assegnazione multipla siano in numero
					minore rispetto a quelli della parte sinistra. Di conseguenza, all'intervallo $var_i \, \dots \, var_k$ corrisponde un solo valore $val_i$ corrispondente
					ad una chiamata a procedura con $k$ valori di ritorno (denotata come \texttt{procName()}).
					In questo caso, il codice generato è $var_i = procName(); var_j = procName\_j \, \forall j \in \{i+1, \dots, k\}$\newline
					Il motivo di questo approccio è descritto nella sezione \ref{cg:procMultipleRet}
		\end{enumerate}
	
	\subsubsection{Gestione della procedura main}
	In fase di progettazione del compilatore, per la gestione della procedura main, si è deciso di permetterne una firma unica
	poiché il linguaggio C, per la funzione main, non prevede valore di ritorno diverso da int.
	In altri termini, la procedura main in Toy sarà sempre nella forma:\newline
	\texttt{proc main() void} \, $vardecl_1 \, \dots \, vardecl_n; \, stat_1, \, \dots \, stat_n;$ \texttt{->corp;} \newline
	Il corrispondente codice C generato per la funzione main (a causa della limitazione introdotta dal linguaggio) sarà sempre nella forma:\newline
	\texttt{int main()}\{$stat_1, \, \dots, \, stat_n;$ \texttt{return 0;}\}
	

	\subsubsection{Gestione del costrutto while}
	Le specifiche sintattiche del linguaggio Toy prevedono le due seguenti produzioni per il costrutto while:
	\begin{enumerate}
		\item 	\texttt{while statlist1 return expr do statlist2 od} 
		\item 	\texttt{while expr do statlist od}
	\end{enumerate}
	In C, il costrutto while corrispondente sarà, rispettivamente:
	\begin{enumerate}
		\item 	\texttt{statlist1 while(expr)\{statlist2; statlist1;\}} 
		\item 	\texttt{while(expr)\{statlist;\}}
	\end{enumerate}
	
	\newpage
    \section{Regole di Type Checking implementate}
    \subsection{Tipi Primitivi}
    \[
        \Gamma \vdash null \colon null \qquad
        \Gamma \vdash true \colon boolean \qquad
        \Gamma \vdash false \colon boolean \qquad
    \]
    \[
        \Gamma \vdash \mi{int} \colon int \qquad
        \Gamma \vdash \mi{float} \colon float \qquad
        \Gamma \vdash \mi{string} \colon string \qquad
        \Gamma \vdash \mi{bool}    \colon boolean \qquad
    \]
    \subsection{Dichiarazioni di Variabili}
    \[
        \frac{(x \colon \tau) \in \Gamma}{\Gamma \vdash x \colon \tau}
    \]
    \subsection{Operazioni Unarie}
    \[
        \frac {\Gamma \vdash \mi{e} \colon \tau_1 \quad optype1(op,\tau_1) = \tau} %
              {\Gamma \vdash (op \, \mi{e}) \colon \tau}
    \]
    \subsection{Operazioni Binarie}
    \[
        \frac {\Gamma \vdash \mi{e_1} \colon \tau_1 \quad \Gamma \vdash \mi{e_2} \colon \tau_2  \quad optype2(op,\tau_1, \tau_2) = \tau} %
        {\Gamma \vdash (\mi{e_1}\, op \, \mi{e_2}) \colon \tau}
    \]
    \subsection{Chiamata a Procedura}
    \[
        \frac {\Gamma \vdash \mi{f} \colon \tau_i^{\,i \, \in \,1 \, \dots \, n} \to \tau_j^{\,j \, \in \, 1 \, \dots \, m} \quad \Gamma \vdash \mi{e_i} \colon %
        \tau_i^{\, i \, \in \, 1 \, \dots \, n}} %
        {\Gamma \vdash \mi{f}(\mi{e}_i^{\, i \, \in \, 1 \, \dots \, n})\colon \tau_j^{\, j \, \in \, 1 \, \dots \, m}}
    \]
    
    \subsection{Statement}
    	\subsubsection{if-then}
	\[
		\frac{\Gamma \vdash \mi{e} \colon boolean \quad \Gamma \vdash \mi{stmt}}%
		{\Gamma \vdash \resw{if} \, \mi{e} \, \resw{then} \, \mi{stmt} \, \resw{fi}}
	\]
	\subsubsection{if-then-else}
	\[
		\frac{\Gamma \vdash \mi{e} \colon boolean \quad \Gamma \vdash \mi{stmt}_{1} \quad \Gamma \vdash \mi{stmt}_{2}}%
		{\Gamma \vdash \resw{if} \, \mi{e} \, \resw{then} \, \mi{stmt}_{1} \, \resw{else} \, \mi{stmt}_{2} \, \resw{fi}}
	\]
	\subsubsection{if-then-elif-else}
	\[
		\frac{\Gamma \vdash \mi{e}_{j}^{\, j \, \in \, 1 \, \dots \, m} \colon boolean \quad %
		\Gamma \vdash \mi{stmt}_{i}^{i \, \in \, 1 \, \dots \, 3}}%
		{%denominatore
		\Gamma \vdash  \resw{if} \, \mi{e}_{1} \, \resw{then} \, \mi{stmt}_{1}  \,
		 (\resw{elif} \, \mi{e}_{j}^{\, j \, \in \, 2 \, \dots \, m} \, \resw{then} \, \mi{stmt}_{2}\,)_{t}^{t \, \in \, 1 \, \dots \, k} \,
		  \resw{else} \, \mi{stmt}_{3} \, \resw{fi}
		}%
	\]
	\subsubsection{while}
	\[
		\frac{\Gamma \vdash \mi{e} \colon boolean \quad \Gamma \vdash \mi{stmt}}%
		{\Gamma \vdash \resw{while} \, \mi{e} \, \resw{do}\, \mi{stmt} \, \resw{od}}
	\]
	\subsubsection{while-return}
	\[
		\frac{\Gamma \vdash \mi{e} \colon boolean \quad \Gamma \vdash \mi{stmt}_{1} \quad \Gamma \vdash \mi{stmt}_{2}}%
		{\Gamma \vdash \resw{while} \, \mi{stmt}_{1}  \resw{-\!\!>}\, \mi{e} \, \resw{do}\, \mi{stmt}_{2} \, \resw{od}}
	\]
	\subsubsection{readln}
	\[
		\frac{(x_{i}^{\, i \, \in \, 1 \, \dots \, n} \colon \tau_{i}^{\, i \, \in \, 1 \, \dots \, n}) \, \in \, \Gamma}%
		{\Gamma \vdash \resw{readln(} \mi{x}_{i}^{\, i \, \in \, 1 \, \dots \, n} \resw{)}}
	\]
    \subsubsection{write}
    \[
        \frac{\Gamma \vdash \mi{e} \colon \tau \quad (\tau \neq \resw{void})\footnotemark}%
        {\Gamma \vdash \resw{write(} \mi{e} \colon \tau \resw{)}}
    \]
    \footnotetext{Si vuole intendere che il tipo di $e$ deve essere diverso da \texttt{void}. Per ulteriori 
    informazioni consultare le scelte di sviluppo del compilatore.}
    \subsubsection{simple-assign}
    \[
        \frac{(\mi{x} \colon \tau) \, \in \, \Gamma \quad \Gamma \vdash \mi{e} \colon \tau}%
        {\Gamma \vdash \mi{x} \, \resw{:=} \, \mi{e}}
    \]
    \subsubsection{multiple-assign}
     \[
        \frac{(\mi{x}_{i}^{\, i \, \in \, 1 \, \dots \, n} \colon \tau_{i}^{\, i \, \in \, 1 \, \dots \, n}) \, \in \, \Gamma \quad \Gamma \vdash \mi{e}_{j}^{\, j \, \in \, 1 \, \dots \, n}% 
        \colon \tau_{j}^{\, j \, \in \, 1 \, \dots \, n}}%
        {\Gamma \vdash \mi{x}_{i}^{\, i \, \in \, 1 \, \dots \, n} \, \resw{:=} \, \mi{e}_{j}^{\, j \, \in \, 1 \, \dots \, n}}
    \]
    \subsubsection{return}
    \[
    	\frac{(\$\mi{ret} \colon \tau) \in \Gamma \quad \Gamma \vdash \mi{e} \colon \tau}%
	{\Gamma \vdash \resw{-\!\!>} \mi{e}}
    \]

	\newpage
    \subsection{Tabelle di Compatibilità}
    \begin{figure}[ht]
    	\centering
	\subfloat[][optype1]{%
    	\begin{tabular}{lll}
    		\toprule
		op & operand & result\\
		\midrule
		- & integer & integer \\
		- & float     & float \\
		! & boolean & boolean \\
		\bottomrule
    	\end{tabular}
    	} \\
	
	\subfloat[][optype2]{%
	\begin{tabular}{llll}
		\toprule
		\multicolumn{1}{c}{\multirow{2}*{op}} & \multicolumn{1}{c}{first}  & \multicolumn{1}{c}{second} & \multicolumn{1}{c}{\multirow{2}*{result}} \\
		                      & \multicolumn{1}{c}{operand} & \multicolumn{1}{c}{operand} \\
		\midrule
		+ - * / & integer & integer & integer \\
		+ - * / & integer & float & float \\
		+ - * / & float & integer & float \\
		+ - * / & float & float & float \\
		\&\& || & boolean & boolean & boolean\\
		< = > <= >= <> & integer & integer & boolean \\
		< = > <= >= <> & integer & float & boolean \\
		< = > <= >= <> & float & integer & boolean \\
		< = > <= >= <> & float & float & boolean \\
		< = > <= >= <> & string & string & boolean \\
		\bottomrule
	\end{tabular}
	}
	\caption{Relazioni di tipo per gli operatori primitivi. Gli operatori aritmetici lavorano sia su numeri interi sia su numeri in virgola mobile.
	              Gli operatori logici ! \&\& || (not, and e or) lavorano su boolean. Gli operatori di comparazione lavorano su tipi primitivi diversi da boolean.}
\end{figure}

\end{document}